\chapter{Reference}
\section{Programs}
\label{sec:ref_programs}
\input{reference/programs/all}
\section{Mapping file}
\label{sec:ref_mapping}
The root node always has to be cg\_molecule. It can contain the following keywords:

\section{Mapping definitions}
{\em Mapping definitions} describe how to map a single molecule from atomistic to coarse-grained representation. The mapping definition have only to be specified once per molecule. The file contains sections for coarse-grained beads, bonded interactions in coarse grained scheme as well as mapping matrices. 

\subsection{Example - mapping file for propane}

\lstinputlisting[frame=single,caption=Mapping for propane]{functionality/propane.xml}


\section{Settings file}
All options for the iterative script are stored in an xml file.
\label{sec:ref_options}
\input{reference/xml/cgoptions.xml}

\subsection{Interaction options}
\label{sec:ref_interaction}
This section contains all interaction option, which could be contained in the \cgopt{non-bonded} or \cgopt{bonded} section in \sect{sec:ref_options}.
\input{reference/xml/cginteraction.xml}
\vfill

\section{Scripts}
\label{sec:csg_table}
Scripts are used by \prog{csg_call} and \prog{csg_inverse}.
The script table commonly used (compare \texttt{csg\_call --list}): 
\input{reference/scripts/csg_table}
Script calls can be overwritten by adding a line with the 3rd column changed to \texttt{csg\_table} in \cgopt{inverse.scriptpath} directory.
\input{reference/scripts/all}
