\section{Mapping definitions}
{\em Mapping definitions} describe how to map a molecule from atomistic to coarse-grained representation. The mapping definition have to be only specified once per molecule. The file contains sections for coarse-grained beads, bonded interactions in coarse grained scheme as well as mapping matrices. 

\subsection{Structure of mapping file}
The root node always has to be cg\_molecule. It can contain the following keywords:

\begin{tabular}{lcl}
  \textbf{name} &  & name of molecule in coarse-grained representation \\
  \textbf{ident} &  & molecule name in atomistic simulations \\
  \textbf{topology} &  & definition of coarse grained topology \\
  \textbf{maps} &  & section containing mapping weights
\end{tabular}

\subsection{Example - mapping file for propane}

\begin{lstlisting}
<cg_molecule>
  <name>ppn</name>
  <ident>ppn</ident>
  <topology>
    <cg_beads>
      <cg_bead>
        <name>A1</name>
        <type>A</type>
        <mapping>A</mapping>
        <beads>1:ppn:C1 1:ppn:H4 1:ppn:H5 1:ppn:H6</beads>
      </cg_bead>

      <cg_bead>
        <name>B1</name>
        <type>B</type>
        <mapping>B</mapping>
        <beads>1:ppn:C2 1:ppn:H7 1:ppn:H8</beads>
      </cg_bead>

      <cg_bead>
        <name>A2</name>
        <type>A</type>
        <mapping>A</mapping>
        <beads>1:ppn:C3 1:ppn:H9 1:ppn:H10 1:ppn:H11</beads>
      </cg_bead>
    </cg_beads>

    <cg_bonded>
      <bond>
        <name>bond</name>
        <beads>
          A1 B1
          B1 A2
        </beads>
      </bond>

      <angle>
        <name>angle</name>
        <beads>
          A1 B1 A2
        </beads>
      </angle>
    </cg_bonded>
  </topology>

  <maps>
    <map>
      <name>A</name>
      <weights> 12 1 1 1 </weights>
    </map>
    <map>
      <name>B</name>
      <weights> 12 1 1 </weights>
    </map>
  </maps>
</cg_molecule>
\end{lstlisting}