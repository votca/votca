\section{Boltzmann Inversion}
Boltzmann inversion is most often performed on a single molecule in vaccuum. VOTCA offers tools to analyse distributions and correlations of such a simulation as well as the preparation of an tabulated potential. The tool most often used in this section is \textbf{csg}. Also mapping from atomistic to coarse-grained level can be perfomred. An important thing to remember if the program \textbf{csg} is used is that it parses the whole trajectory and stores all information about bonded interactions in memory to allow interactive analysis. That's why it is not suitable for really big systems with lots of chains. If analysis of a big system is performed, \textbf{csg\_stat} should be prefered for the analysis. Though it currently offers less features than \textbf{csg}, memory problems should not occur.

\subsection{Mapping scheme}
The first thing to coarse-grain a system is to define a mapping scheme (see \sect{sec:mapping}). The mapping is defined by a simple xml file for each molecule-type. An important step is to verify the mapping scheme by:

\begin{verbatim}
  csg --top topol.tpr --trj traj.trr --cg cg.xml --out cg.gro
\end{verbatim}

The most common thing that can go wrong is that beads cannot be found. On reason is a type in the atom name of the mapping. To debug which atoms are read in from a topology file, \textbf{csg\_dump} can be used. A second reason could be, that molecules are not identified correctly. In case of a multicomponent system, see \sect{sec:adv_topology} for details.

In case everything works find, a file cg.gro which contains the coarse-grained trajectory is created. The easuiest way to compare coarse-grained and atomistic representation is to open both in e.g. vmd.  Be careful when opening in vmd specifying a .gro file + .trr, the first frame comes from the .gro, the following correspond to the trr. The coarse grained file only contains the frames from the trajectory, to compare them, the first frame of the atomistic run has to be deleted in vmd!!

\subsection{Generating trajectories}
Write about excludion lists!

\subsection{Analysis}
When \textbf{csg} finished to parse the trajectory, it enters an interactive mode which accepts commands. See reference for details.
