\chapter{Inverse Monte Carlo}
In this section, additional options are described to run \imc coarse graining. The usage of imc is similar to \ibi, it is necessary to understand the use of the scripting framework described in chapter~\ref{sec:iterative_workflow}.

\textbf{WARNING: multicomponent \imc is still experimental!}

\section{General considerations}
In comparison to \ibi, \imc needs a significant amount more statistics to calculate the potential update\cite{Ruehle:2009.a}. It is advisable to perform smoothing on the potentual update. Smoothing can can be performed as described in \sect{ref:ibi:optimize}. In addition, \imc can lead to problems with finite size. For methanol, it was shown, that a too small system leads to a linear shift in the potential\cite{Ruehle:2009.a}. Always check, that the system size is sufficient and runlengh csg smoothing iterations is well balanced.

\section{Additional mapping for statistics}
The program \prog{csg_stat} is used for evaluating the \imc matrix. Although here it only acts on the coarse-grained system, it still needs a mapping file to work. This will change with one of the next releases to simplify the setup. The mapping file needs to be a one to one mapping of the coarse grained system, e.g. for coarse graining \spce water, the mapping file looks like the following.
\begin{lstlisting}
  </cg_molecule>
    <name>SOL</name> 
    <ident>SOL</ident>
    <topology>
      <cg_beads>
        <cg_bead>
          <name>CG</name>
          <type>CG</type>
          <mapping>A</mapping>
          <beads>
            1:SOL:CG 
          </beads>
        </cg_bead>
      </cg_beads>
    </topology>
    <maps>
      <map>
        <name>A</name>
        <weights>1</weights>
      </map>
    </maps>
  </cg_molecule>
\end{lstlisting}



\section{Correlation groups}
Unlike in \ibi, \imc also takes into account cross-correlations of interactions to calculate the update. However, it might not always be beneficial to evaluate cross-correlations of all pairs of interactions. By specifying \interopt{imc.group}, \votca allows to define groups of interactions, amongst which cross-correlations are taken into account, where \interopt{imc.group} can be any name.

\begin{lstlisting}
  <non-bonded>
    <name>CG-CG</name>
    <type1>CG</type1>
    <type2>CG</type2>
    ...
    <imc>
      <group>solvent</group>
   </imc>
  </non-bonded>
  <non-bonded>
\end{lstlisting}
