\section{Transfer integrals }
\label{sec:transfer_integrals}

The electronic transfer integral\index{electronic coupling}\index{transfer integral|see{electronic coupling}} element $J_{ij}$ entering the Marcus rates in \equ{marcus} is defined as
\begin{equation}
   J_{ij} = \left\langle \phi_i \left\vert \hat{H} \right\vert \phi_j \right\rangle ,
\label{equ:TI}
\end{equation}
where $\phi_i$ and $\phi_j$ are diabatic wavefunctions, localized on molecule $i$ and $j$ respectively, participating in the charge transfer, and $\hat{H}$ is the Hamiltonian of the formed dimer. Within the frozen-core approximation, the usual choice for the diabatic wavefunctions $\phi_i$\index{diabatic states} is the highest occupied molecular orbital (HOMO) in case of hole transport, and the lowest unoccupied molecular orbital (LUMO) in the case of electron transfer, while $\hat{H}$ is an effective single particle Hamiltonian, e.g. Fock or Kohn-Sham operator of the dimer. As such, $J_{ij}$ is a measure of the strength of the electronic coupling of the frontier orbitals of monomers mediated by the dimer interactions. 

Intrinsically, the transfer integral is very sensitive to the molecular arrangement, i.e. the distance and the mutual orientation of the molecules participating in charge transport. Since this arrangement can also be significantly influenced by static and/or dynamic disorder~\cite{bassler_charge_1993,troisi_charge-transport_2006,troisi_charge_2009,mcmahon_organic_2010,vehoff_charge_2010},
it is essential to calculate $J_{ij}$ explicitly for each hopping pair within a realistic morphology. Considering that the number of dimers for which \equ{TI} has to be evaluated is proportional to the number of molecules times their coordination number, computationally efficient and at the same time quantitatively reliable schemes are required.

\subsection{Projection of monomer orbitals on dimer orbitals (DIPRO)}
\label{sec:dipro}
An approach for the determination of the transfer integral that can be used for any single-particle electronic structure method (Hartree-Fock, DFT, or semiempirical methods) is based on the projection of monomer orbitals on a manifold of explicitly calculated dimer orbitals. This dimer projection (DIPRO) technique including an assessment of computational parameters such as the basis set, exchange-correlation functionals, and convergence criteria is presented in detail in Ref.~\cite{baumeier_density-functional_2010}. A brief summary of the concept is given below.

We start from an effective Hamiltonian~\footnote{we use following notations: $a$ - number, $\vctr{a}$ - vector, $\matr{A}$ - matrix, $\oper{A}$ - operator}
%
\begin{equation}
  \oper{H}^\text{eff} = \sum_i \epsilon_i \oper{a}_i^\dagger \oper{a}_i + \sum_{j \neq i} J_{ij} \oper{a}_i^\dagger \oper{a}_j + c.c.
  \label{equ:dipro_eq1}
\end{equation}
%
where $\oper{a}_i^\dagger$ and $\oper{a}_i$ are the creation and annihilation operators for a charge carrier located at the molecular site $i$.
The electron site energy is given by $\epsilon_i$, while $J_{ij}$  is the transfer integral between two sites $i$ and $j$. We label their frontier orbitals (HOMO for hole transfer, LUMO for electron transfer) $\phi_i$ and $\phi_j$, respectively. Assuming that the frontier orbitals of a dimer (adiabatic energy surfaces) result exclusively from the interaction of the frontier orbitals of monomers, and consequently expand them in terms of $\phi_i$ and $\phi_j$. The expansion coefficients, $\vctr{C}$, can be determined by solving the secular equation
%
\begin{equation}
  (\matr{H} - E \matr{S})\vctr{C} = 0
  \label{equ:dipro_eq2}
\end{equation}
%
where $\matr{H}$ and $\matr{S}$ are the Hamiltonian and overlap matrices of the system, respectively. 
%
%Since it is easier to work in matrix form, the following
%equation also holds (equation (\ref{eq:dipro_eq2}) in matrix form):
%
%\begin{equation}
% \matr{H}\matr{U} = \matr{S}\matr{U}\matr{E}
%  \label{eq:dipro_eq3}
% \end{equation}
%
These matrices can be written explicitly as
%
\begin{equation}
% \begin{aligned}
  \matr{H} = 
  \begin{pmatrix}
    e_i    &  H_{ij} \\
    H_{ij}^* &  e_j  
  \end{pmatrix} \hspace{2cm}
  \matr{S} = 
  \begin{pmatrix}
    1    &  S_{ij} \\
    S_{ij}^* &  1  
  \end{pmatrix}
%  \end{aligned}
  \label{equ:dipro_eq3}
\end{equation}
%
with 
%
\begin{equation}
 \begin{aligned}
  e_i &= \Bra{\phi_i}\oper{H} \Ket{\phi_i} \hspace{2cm}  H_{ij} = \Bra{\phi_i}\oper{H} \Ket{\phi_j}\\
  e_j &= \Bra{\phi_j}\oper{H} \Ket{\phi_j} \hspace{2cm}  S_{ij} = \Bra{\phi_j} \phi_j\rangle %S 
 \end{aligned}
  \label{equ:dipro_eq4}
\end{equation}
The matrix elements $e_{i(j)}$, $H_{ij}$, and $S_{ij}$ entering \equ{dipro_eq3} can be calculated via projections on the dimer orbitals (eigenfunctions of $\hat{H}$) $\left\{\Ket{\phi^\text{D}_n}\right\}$ by inserting $\oper{1} = \sum_n \Ket{\phi^\text{D}_n}\Bra{\phi^\text{D}_n}$ twice. We exemplify this explicitly for $H_{ij}$ in the following
%
\begin{equation}
  H_{ij} = \sum_{nm}{\Braket{\phi_i|\phi^\text{D}_n} \Bra{\phi^{D}_n}\hat{H}\Ket{\phi^\text{D}_m}\Braket{\phi^\text{D}_m|\phi_j}} .
  \label{eq:dipro_eq16}
\end{equation}
%
The Hamiltonian is diagonal in its eigenfunctions, $\Bra{\phi^\text{D}_n}\oper{H}\Ket{\phi^\text{D}_m} = E_n \delta_{nm}$. Collecting the projections of the frontier orbitals  $\Ket{\phi_{i(j)}}$ on the $n$-th dimer state $\left(\vctr{V}_{(i)}\right)_n= \Braket{\phi_i|\phi^\text{D}_n}$ and $\left(\vctr{V}_{(j)}\right)_n=\Braket{\phi_j|\phi^\text{D}_n}$ respectively, into vectors we obtain

\begin{equation}
   H_{ij} = \vctr{V}_{(i)} \matr{E}   \vctr{V}_{(j)}^\dagger .
  \label{eq:dipro_eq17}
\end{equation}
%
What is left to do is determine these projections $\vctr{V}_{(k)}$. In all practical calculations the molecular orbitals are expanded in basis sets of either plane waves or of localized atomic orbitals $\Ket{\varphi_\alpha}$. We will first consider the case that the calculations for
the monomers are performed using a counterpoise basis set that is commonly used to deal with the basis set superposition error (BSSE). The basis set of atom-centered orbitals of a monomer is extended to the one of the dimer by adding the respective atomic orbitals at virtual coordinates of the second monomer. We can then write the respective expansions as

\begin{equation}
 %\begin{aligned}
  \Ket{\phi_{k}} = \sum_{\alpha} \lambda^{(k)}_\alpha \Ket{\varphi_\alpha} \hspace{1cm}\text{and}\hspace{1cm}
  \Ket{\phi^\text{D}_n} = \sum_{\alpha} D^{(n)}_\alpha \Ket{\varphi_\alpha}
  \label{eq:dipro_eq18}
\end{equation}
%
where $k=i,j$. The projections can then be determined within this common basis set as

 \begin{equation}
  \begin{aligned}
     \left(\vctr{V}_k\right)_n=\Braket{\phi_k|\phi^\text{D}_n} = \sum_{\alpha} \lambda^{(k)}_{\alpha} \Bra{\alpha} \sum_{\beta} D^{(n)}_{\beta} \Ket{\beta} = 
     \vctr{\boldsymbol{\lambda}}_{(k)}^\dagger \matr{\mathcal{S}} \vctr{D}_{(n)} 
%     %\\
% %    \Braket{B|i} = \sum_{\alpha} B_{\alpha} \Bra{\alpha}
% %    \sum_{\beta} D^{(i)}_{\beta} \Ket{\beta} = 
% %    \vctr{B}^\dagger \matr{S} \vctr{D}^{(i)} \\
  \end{aligned}
   \label{eq:dipro_eq19}
 \end{equation}
where $\matr{\mathcal{S}}$ is the overlap matrix of the atomic basis functions. This allows us to finally write the elements of the Hamiltonian and overlap matrices in \equ{dipro_eq3} as:

 \begin{equation}
  \begin{aligned}
     H_{ij} &= \vctr{\boldsymbol{\lambda}}_{(i)}^\dagger \matr{\mathcal{S}} \matr{D} \matr{E} \matr{D}^\dagger \matr{\mathcal{S}}^\dagger \vctr{\boldsymbol{\lambda}}_{(j)}  \\
     S_{ij} &= \vctr{\boldsymbol{\lambda}}_{(i)}^\dagger \matr{\mathcal{S}} \matr{D}  \matr{D}^\dagger \matr{\mathcal{S}}^\dagger \vctr{\boldsymbol{\lambda}}_{(j)} 
  \end{aligned}
   \label{eq:dipro_eq20}
 \end{equation}
%
Since the two monomer frontier orbitals that form the basis of this expansion are not orthogonal in general ($\matr{S} \neq \matr{1}$), it is necessary to transform \equ{dipro_eq2} into a standard eigenvalue problem of the form
%
\begin{equation}
  \matr{H}^{\mathrm{eff}} \vctr{C}^{\mathrm{eff}} =   E \vctr{C}^{\mathrm{eff}} 
  \label{eq:dipro_eq7}
\end{equation}
%
to make it correspond to \equ{dipro_eq1}. According to L\"owdin such a transformation can be achieved by
%
\begin{equation}
  \matr{H^\mathrm{eff}} = \matr{S}^{\left. {-1} \middle/ {2} \right.}
  \matr{H}\matr{S}^{\left. {-1} \middle/ {2} \right.}.
  \label{eq:dipro_eq9}
\end{equation}
%
This then yields an effective Hamiltonian matrix in an orthogonal basis, and its entries can directly be identified with the site energies $\epsilon_i$ and transfer integrals $J_{ij}$:
%
\begin{equation}
 \begin{aligned}
  \matr{H}^{\mathrm{eff}} &= 
    \begin{pmatrix}
      e_i^{\mathrm{eff}}    &  H_{ij}^\mathrm{eff} \\
      H_{ij}^{*,\mathrm{eff}}   &  e_j^\mathrm{eff}  
    \end{pmatrix} =
    \begin{pmatrix}
      \epsilon_i    &  J_{ij} \\
      J_{ij}^*      &  \epsilon_j  
    \end{pmatrix} 
 \end{aligned}
  \label{eq:dipro_eq11}
\end{equation}

 \begin{figure}[htb]
     \center
     \includegraphics[width=\linewidth]{fig/idft_flow/schemes_all}
     \caption{Schematics of the DIPRO method. (a) General workflow of the projection technique. (b) Strategy of the efficient noCP+noSCF implementation, in which the monomer calculations are performed independently form the dimer configurations (noCP), using the \calc{edft} \calculator. The dimer Hamiltonian is subsequently constructed based on an initial guess formed from monomer orbitals and only diagonalized once (noSCF) before the transfer integral is calculated by projection. This second step is performed by the \calc{idft} \calculator. }
     \label{fig:dipro_scheme}
 \end{figure}



\subsection{Transfer integrals: \gaussian and \turbomole}

Apartm from semi-empirical methods, we also provide interfaces for a DFT-based evaluation of electronic coupling elements.

\begin{itemize}
\item {\it not sure about directory structure yet, using {\tt
      \$DIRECTORY} for the time being}
\item creating file structure for frame $N$ (raw: no postpocessing,
  min: MD energy minimization) in directory {\tt OUTDIR}
\begin{verbatim}
$DIRECTORY/perpare.sh raw/min N OUTDIR
cd OUTDIR
$DIRECTORY/pairdump.sh
\end{verbatim}
\item make sure QCP environments are set!
\item running calculations for all monomers
 \begin{verbatim}
$DIRECTORY/calc_monomer QCP [METHOD]

QCP:   G for Gaussian09
       T for Turbomole

METHOD: func/basis (optional)
        overrides default functional/basisset combination
        defaults: pbepbe/6-311G** Gaussian09
                  b-p/def-TZVP    Turbomole
\end{verbatim}
\item check monomer calculations (Gaussian version to test!) 
\begin{verbatim}
$DIRECTORY/check_mols N M QCP

N:   First monomer to test
M:   Last monomer to test
QCP: G/T 
\end{verbatim}
incomplete monomers are written to file {\tt TROUBLE.mol}
\item running calculations for all dimers
 \begin{verbatim}
$DIRECTORY/calc_dimer_noSCF QCP [METHOD]

QCP:   G for Gaussian09
       T for Turbomole

METHOD: func/basis (optional)
        overrides default functional/basisset combination
        defaults: pbepbe/6-311G** Gaussian09
                  b-p/def-TZVP    Turbomole
\end{verbatim}
\item should we add {\tt trajectory\_submit.sh} that does all monomer
  and dimer calculations on the cluster (MPIP-specific)?
\end{itemize}


\section{Semi-empirical determination of transfer intergals}
\label{sec:moo}

\newcommand{\xyz}{\texttt{geometry.xyz}\xspace}
\newcommand{\orb}{\texttt{zindo.orb}\xspace}

The main purpose of the molecular orbital overal library (MOO) is fast evaluation of electronic coupling elements. It is based on the semi-empirical ZINDO Hamiltonian and therefore has limited applicability. The general advice is to first compare the accuracy of this method to the DFT-based calculations. MOO constructs the Fock operator of a dimer from the  molecular orbitals of monomers by translating and rotating the orbitals. Hence, it requires the optimized geometry of the molecule (\xyz) and the projection coefficients of the molecular on atomic orbitals (\orb). 

\xyz file contains four columns, first being the atom type and the next three its coordinates. The \orb can be generated using \gaussian program and the input script \texttt{get\_orbitals.com} which shown in listing~\ref{list:zindo_orbitals}.

\lstinputlisting[
 label=list:zindo_orbitals, 
 caption={\small \gaussian input file \texttt{get\_orbitals.com} used for generating molecular orbitals. The first line contains  the name of the check file, the second the requested RAM. 
%
 \texttt{int=zindos} requests the method ZINDO, \texttt{punch=mo} states that the molecular orbitals ought to be written to  the \texttt{fort.7} file, \texttt{nosymm} forbids use of symmetry and is necessary to ensure correct position of orbitals with respect to the provided coordinates. The two integer numbers correspond to the charge and multiplicity of the system: $0\, 1$ corresponds to a neutral system with a multiplicity of one. They are followed by the types and coordinates of all atoms in the molecule.
}]%
{./fig/moo/get_orbitals.com}

Provided with this input, \gaussian will generate \texttt{fort.7} file containing the molecular orbitals of a single molecule. This file can be renamed to \orb. 

\subsection{Conjugated segments}

Decription of conjugated segments is stored in \texttt{charges.xml}.

\lstinputlisting[
 label=list:conjugated_segments, 
 caption={\small \xml file describing conjugated segments.
}]%
{./fig/moo/charges.xml}


{\color{red} Old text from Thorsten. Cleaning is needeed.}



\begin{itemize}
 \item {\bf posname} \\
 Location of {\bf INPUT\_COORDS}.
 \item {\bf orbname} \\
 Location of {\bf fort.7}.
 \item{\bf basis} \\
 This should be set to INDO, unless the fort.7 has been created using another basis set. In that case it must be set to an xml file setting the characteristics of the basis set.
 \item {\bf transorb} \\
 Number of HOMO (LUMO) orbital. Corresponds to the number of $\alpha$ electrons in the \emph{Gaussian} log-file {\bf get\_orbitals.log} minus one (since counting in C++ starts at zero) for the HOMO and the number of $\alpha$ electrons for the LUMO.
 \item {\bf reorg} \\
 Reorganization energy of the cation or anion in eV calculated via \emph{Gaussian}. mp2 should be used at least for anions.
 \item {\bf name} \\
 Name of the mapping of the molecule. Must correspond to CG mapping.
 \item {\bf energy} \\
 Energy of the HOMO/LUMO level
 \item {\bf monomer\_atom\_map} \\
 List of atom indices as they were specified in the \emph{Gaussian} input used to create the {\bf fort.7} file. \\
 Note: The first three values are important, since they must correspond to the first three atoms defined in the coarse-grained mapping, which are used to calculate two vectors indicating the orientation of the molecule. The third required vector is the eigenvector of the smallest eigenvalue of the gyration tensor, i.e. perpendicular to the planar core. \\
 Note: The number of molecules here may differ from that in the coarse-grained mapping, since for example only the core is important for transport and not the side chains, but it has to be the same number of atoms as in the \emph{Gaussian} input file otherwise overlap integral values will be terribly wrong.
\end{itemize}

